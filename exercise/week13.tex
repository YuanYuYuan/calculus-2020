\documentclass{article}
\usepackage{amsmath}
\usepackage{amsfonts}
\usepackage{amssymb}
\usepackage{mathtools}
\usepackage{tcolorbox}
\usepackage[inline]{enumitem}
\usepackage[a4paper,margin=1in]{geometry}
\usepackage[normalem]{ulem}
\usepackage{graphicx}
\usepackage{tasks}
\settasks{label=(\alph*), label-offset=0.4em, label-width=1.5em}

\usepackage{fancyhdr}
\fancyhf{}
\setlength{\headheight}{36pt}
\renewcommand{\headrulewidth}{0pt}
\thispagestyle{fancy}
\lhead{Calculus Exercise}
\chead{Week 13 (7.4, 7.5, 7.8)}
\rhead{\underline{ID:\hspace{7.4em}} \\ \vspace{0.2cm} \uline{Name:\hspace{6em}}}
\cfoot{\thepage}

\begin{document}
\begin{enumerate}
\item[7.4.27]
    Evaluate the integral.
    \[
        \int \frac{4x}{x^{3} + x^{2} + x + 1}dx
    \]


\vspace{6cm}

\item[7.4.50]
    Make a substitution to express the integrand as a rational function and then
    evaluate the integral.
    \[
        \int \frac{\sqrt{1 + \sqrt{x}}}{x} dx
    \]

\vspace{6cm}

\item[7.4.58]
    Use integration by parts, together with the techniques of this section,
    to evaluate the integral.
    \[
        \int x \tan^{-1} x dx
    \]

\newpage

\item[7.4.60]
    Evaluate $\displaystyle \int \frac{dx}{x^{2} + k}$ by considering several
    cases for the constant $k$.

\vspace{8cm}

\item[7.4.63]
    \textbf{Weierstrass Substitution} The German mathematical Karl Weierstrass
    (1815 - 1897) noticed that the substitution $t=\tan (\frac{x}{2})$
    will convert any rational function of $\sin x$ and $\cos x$ into an
    ordinary rational function of $t$.

    \begin{enumerate}
        \item If $t=\tan (\frac{x}{2})$, $-\pi < x < \pi$, sketch a right
            triangle or use trigonometric identities to show that
            \[
                \cos \frac{x}{2} = \frac{1}{\sqrt{1+t^{2}}}
                \text{ and }
                \sin \frac{x}{2} = \frac{t}{\sqrt{1+t^{2}}}.
            \]
        \item Show that
            \[
                \cos x = \frac{1-t^{2}}{\sqrt{1+t^{2}}}
                \text{ and }
                \sin x = \frac{2t}{\sqrt{1+t^{2}}}.
            \]
        \item Show that
            \[
                dx = \frac{2}{1+t^{2}}dt.
            \]
    \end{enumerate}

\newpage

\item[7.5.8]
    Three integrals are given that, although they look similar, may
    require different techniques of integration. Evaluate the integrals.

    (a) $\displaystyle \int e^{x} \sqrt{e^{x}-1}dx$
    \hspace{1cm}
    (b) $\displaystyle \int \frac{e^{x}}{\sqrt{1-e^{2x}}}dx$
    \hspace{1cm}
    (c) $\displaystyle \int \frac{1}{\sqrt{e^{x}-1}}dx$

\vspace{6cm}

\item[7.5.27]
    Evaluate the integral $\displaystyle \int e^{x+e^{x}}dx$.

\vspace{6cm}

\item[7.5.44]
    Evaluate the integral $\displaystyle \int \frac{1+\sin x}{1+\cos x}dx$.

\newpage

\item[7.5.76]
    Evaluate the integral $\displaystyle \int \frac{x^{2}}{x^{6}+3x^{3}+x}dx$.

\vspace{5cm}

\item[7.5.93]
    Evaluate the integral
    $\displaystyle \int_{0}^{\frac{\pi}{6}}  \sqrt{1 + \sin 2\theta}d\theta$.

\vspace{5cm}

\item[7.8.68]
    \textbf{Improper Integrals that Are Both Type 1 and Type 2}

    The integral $\displaystyle \int_{a}^{\infty} f(x) dx $ is improper because
    the interval $[a, \infty)$ is infinite. If $f$ has an infinite
    discontinuity at $a$, then the integral is improper for a second
    reason. In this case we evaluate the integral by expressing it as a sum
    of improper integrals of Type 2 and Type 1 as follows:
    \[
        \int_{0}^{\infty} f(x) dx
        = \int_{a}^{c} f(x)dx + \int_{c}^{\infty} f(x)dx,\ c > a
    \]
    Evaluate the given integral if it is convergent.
    \[
        \int_{2}^{\infty} \frac{1}{x \sqrt{x^{2}-4}}dx
    \]

\newpage

\item[7.8.74]
    The \textit{average speed} of molecules in an ideal gas is
    \[
        \bar{v} = \frac{4}{\sqrt{\pi}} \left( \frac{M}{2RT} \right)^{\frac{3}{2}}
        \int_{0}^{\infty} v^{3}e^{-\frac{Mv^{2}}{2RT}} dv,
    \]
    where $M$ is the molecular weight of the gas, $R$ is the gas constant,
    $T$ is the gas temperature, and $v$ is the molecular speed.

    Show that
    \[
        \bar{v} = \sqrt{\frac{8RT}{\pi M}}.
    \]

\vspace{5cm}

\item[7.8.75]
    We know that the region
    $R = \{(x, y) | x \geqslant 1, 0 \leqslant y \leqslant \frac{1}{x}\}$
    has infinite area. Show that by rotating $R$ about the $x$-axis
    we obtain a solid (called \textit{Gabriel's horn}) with finite volume.

\vspace{5cm}

\item[7.8.80]
    A radioacitve substance decays exponentially: The mass at time $t$
    is $m(t) = m(0) e^{kt}$, where $m(0)$ is the initial mass and
    $k$ is a negative constant. The \textit{mean life} $M$ of an atom
    in the substance is
    \[
        M = -k \int_{0}^{\infty} t e^{kt} dt.
    \]

    For the radioactive carbon isotope, $\prescript{14}{}{C}$, used
    in radiocarbon dating, the value of $k$ is -0.000121.
    Find the mean life of a $\prescript{14}{}{C}$ atom.

\end{enumerate}
\end{document}
