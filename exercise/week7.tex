\documentclass{article}
\usepackage{amsmath}
\usepackage{amssymb}
\usepackage{amsfonts}
\usepackage{tcolorbox}
\usepackage[inline]{enumitem}
\usepackage[a4paper,margin=1in]{geometry}
\usepackage[normalem]{ulem}
\usepackage{graphicx}
\usepackage{tasks}
\settasks{label=(\alph*), label-offset=0.4em, label-width=1.5em}

\usepackage{fancyhdr}
\fancyhf{}
\setlength{\headheight}{36pt}
\renewcommand{\headrulewidth}{0pt}
\thispagestyle{fancy}
\lhead{Calculus Exercise}
\chead{Week 7 (4.2, 4.3, 4.4)}
\rhead{\underline{ID:\hspace{7.4em}} \\ \vspace{0.2cm} \uline{Name:\hspace{6em}}}
\cfoot{\thepage}

\begin{document}
\begin{enumerate}
\item[4.2.23]
    Show that the equation has exactly one real solution.
    \[
        2x + \cos (x) = 0
    \]

\vspace{5cm}

\item[4.2.28]
    \begin{enumerate}
        \item
            Suppose that $f$ is differentiable on $\mathbb{R}$ and has two zeros.
            Show that $f'$ has at least one zero.
        \item
            Suppose $f$ is twice differentiable on $\mathbb{R}$ and
            has three zeros. Show that $f''$ has at least one
            real zero.
        \item
            Can you generalize parts (a) and (b)?
    \end{enumerate}

\vspace{6cm}

\item[4.2.35]
    Use the Mean Value Theorem to prove the inequality
    \[
        | \sin (a) - \sin (b) | \leqslant | a - b |,\ \forall\ a, b \in \mathbb{R}.
    \]

\newpage

\item[4.2.39]
    Use the method of \textbf{Example 6} to prove the identity.
    \[
        2 \sin^{-1} (x) = \cos^{-1} (1 - 2x^{2}),\ x \geqslant 0
    \]

    \begin{tcolorbox}
        \textbf{Example 6} Prove the identity $ \displaystyle  \tan^{-1} (x) + \cot^{-1} (x) = \frac{\pi}{2}$.

        Solution. If $f(x) = \tan^{-1} (x) + \cot^{-1} (x)$, then
        \[
            f'(x) = \frac{1}{x+x^{2}} - \frac{1}{1+x^{2}} = 0, \forall\ x
        \]

        Therefore $f(x)  \equiv C$, for some $C \in \mathbb{R}$. And we can find $C = f(1) = \frac{\pi}{2}$.
    \end{tcolorbox}

\vspace{8cm}

\item[4.2.42]
    \textbf{Fixed Points} A number $a$ is called a \textit{fixed point}
    of a function $f$ if $f(a)=a$. Prove that if $f'(x) \neq 1$
    for all real numbers $x$, then $f$ has at most one
    fixed point.

\newpage

\item[4.3.55]
    \begin{enumerate}
        \item Find the intervals of increase or decrease.
        \item Find the local maximum and minimum values.
        \item Find the intervals of concavity and the inflection points.
        \item Use the information from parts (a)-(c) to sketch the graph.
    \end{enumerate}
    \[
        f(x) = x^{\frac{1}{3}} (x + 4)
    \]

\vspace{8cm}

\item[4.3.62]
    \begin{enumerate}
        \item Find the vertical and horizontal asymptotes.
        \item Find the intervals of increase or decrease.
        \item Find the local maximum and minimum values.
        \item Find the intervals of concavity and the inflection points.
        \item Use the information from parts (a)-(d) to sketch the graph of $f$.
    \end{enumerate}
    \[
        f(x) = \frac{e^{x}}{1-e^{x}}
    \]

\newpage

\item[4.3.64]
    Answer the problem stated in 4.3.62 with
    \[
        f(x) = \frac{e^{x}}{1-e^{x}}
    \]

\vspace{8cm}

\item[4.3.84]
    For what values of the numbers $a$ and $b$ dose the function
    \[
        f(x) = axe^{bx^{2}}
    \]
    have the maximum value $f(2) = 1$?

\newpage

\item[4.3.99]
    The three cases in the First Derivative Test cover the situations
    commonly encountered but do not exhaust all possibilities.
    Consider the functions $f,g$, and $h$ whose values at $0$ are all
    $0$ and, for $x \neq 0$,
    \[
        f(x) = x^{4} \sin \frac{1}{x} ,\
        g(x) = x^{4} \left( 2+\sin \frac{1}{x}  \right),\
        h(x) = x^{4} \left( -2 + \sin \frac{1}{x}  \right)
    \]

    \begin{enumerate}
        \item Show that $0$ is a critical number of all three functions
            but their derivatives change sign infinitely often on
            both sides of $0$.
        \item Show that $f$ has neither a local maximum nor a
            local minimum at $0$, $g$ has a local minimum, and
            $h$ has a local maximum.
    \end{enumerate}

\vspace{8cm}

\item[4.4.60]
    Find the limit
    \[
        \lim_{x \to \infty} \left( 1+\frac{a}{x} \right)^{bx}
    \]

\vspace{4cm}


\item[4.4.69]
    Find the limit
    \[
        \lim_{x \to 0^{+}} \frac{x^{x}-1}{\ln x + x - 1}
    \]

\newpage

\item[4.4.76]
    Prove that
    \[
        \lim_{x \to \infty} \frac{\ln x}{x^{p}} = 0
    \]
    for any number $p>0$. This shows that the logarithmic function
    approaches infinity more slowly than any power of $x$.

\vspace{6cm}

\item[4.4.78]
    What happens if you try to use l'Hospital's Rule to find the limit?
    Evaluate the limit using another method.
    \[
        \lim_{x \to \frac{\pi}{2}^{-}}  \frac{\sec (x) }{\tan (x) }
    \]
\vspace{6cm}

\item[4.4.90]
    For what values of $a$ and $b$ is the following equation true?
    \[
        \lim_{x \to 0} \left( \frac{\sin (2x) }{x^{3}} + a + \frac{b}{x^{2}} \right) = 0
    \]
\end{enumerate}
\end{document}
