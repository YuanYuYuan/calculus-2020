\documentclass{article}
\usepackage{amsmath}
\usepackage{amsfonts}
\usepackage{amssymb}
\usepackage{tcolorbox}
\usepackage[inline]{enumitem}
\usepackage[a4paper,margin=1in]{geometry}
\usepackage[normalem]{ulem}
\usepackage{graphicx}
\usepackage{tasks}
\settasks{label=(\alph*), label-offset=0.4em, label-width=1.5em}

\usepackage{fancyhdr}
\fancyhf{}
\setlength{\headheight}{36pt}
\renewcommand{\headrulewidth}{0pt}
\thispagestyle{fancy}
\lhead{Calculus Exercise}
\chead{Week 10 (5.4, 5.5)}
\rhead{\underline{ID:\hspace{7.4em}} \\ \vspace{0.2cm} \underline{Name:\hspace{6em}}}
\cfoot{\thepage}

\begin{document}
\begin{enumerate}
\item[5.4.22] Find the general indefinite integral.
    \[
        \int \sec t (\sec t + \tan t  ) dt
    \]

\vspace{6cm}

\item[5.4.54] Evaluate the definite integral.
    \[
        \int_{0}^{ \frac{3\pi}{2}}  | \sin x | dx
    \]

\vspace{6cm}

\item[5.4.72]
    The acceleration function $a(t)$ in m/s$^2$ and the initial velocity $v(0)$ are given for a particle moving
    along a line. Find (a) the velocity at time $t$ and (b) the distance traveled during the given time
    interval.
    \[
        a(t) = 2t + 3,\ v(0) = -4,\ 0 \leqslant t \leqslant 3.
    \]

\newpage

\item[5.4.77] The marginal cost of manufacturing $x$ yards of a certain fabric is
    \[
        C'(x) = 3 - 0.01x + 0.000006x^{2}
    \]
    (in dollars per yard). Find the increase in cost if the production level
    is raised from $2000$ yards to $4000$ yards.

\vspace{6cm}

\item[5.5.80]
    Evaluate the definite integral.
    \[
        \int_{1}^{16} \frac{x^{\frac{1}{2}}}{1+x^{\frac{3}{4}}} dx
    \]

\vspace{6cm}

\item[5.5.83]
    Evaluate
    \[
        \int_{-2}^{2} (x+3)\sqrt{4-x^{2}}dx
    \]
    by writing it as a sum of two integrals and interpreting one of those
    integrals in terms of an area.

\newpage

\item[5.5.94]
    If $f$ is continuous and $\displaystyle \int_{0}^{9} f(x) dx = 4$, find
    $\displaystyle  \int_{0}^{3} xf(x^{2}) dx$.

\vspace{6cm}

\item[5.5.98]
    If $f$ is continuous on $[0, \pi ]$, use the substitution $u = \pi - x$
    to show that
    \[
        \int_{0}^{\pi} xf(\sin x) dx = \frac{\pi}{2} \int_{0}^{\pi} f(\sin x ) dx.
    \]

\vspace{6cm}

\item[5.5.99]
    Use 5.5.98 to evaluate the integral
    \[
        \int_{0}^{\pi} \frac{x \sin x }{1 + \cos^{2} x }dx.
    \]
\end{enumerate}
\end{document}
